\documentclass{article} \usepackage[utf-8]{inputenc} \begin{document}\section{Com a prenda do padrinho}

X: 1
M: 4/4
K: C
L: 1/8
E3 E EE FG | D2 D2 DD EF | C2 C4 D2 | E2 C6 
w: Com a pren-da do pa-dri-nho seis lá-pis a-go-ra, eu te-nho
|: A2 AG  AG Ac | G2 G4 FE |1 D3 E F2 EF| G2 G4 z2 :|2 D3 E F2 ED| C2 C2 z2 ||
w: ver-de~a-ma-re-lo~a-zul-ma-ri-nho, en-car-na-do, pre-to~e cas-ta-nho, pre-to~e cas-ta-nho
G,2| E3 E  EC FE | D2 D4 DE | F3 F F2 GF | E4 z2
w: De ver-de pin-tei a cam-pi-na sor-ri-den-te~ao sol da ma-nha
|:CC | A3 B  cc BA | G2 G4 FE |1 D2 DE F2 EF | G6 :|2 D2 DE F2 ED | C6 |]
w: a-ma-re-la~e-ra~a flor pe-que-ni-na qu~eu co-lhi pa-ra dar a ma-ma dar a ma-ma
%\includegraphics...
\\
{soc}\\
Com a prenda do padrinho\\
6 lápis agora, eu tenho\\
(verde, amarelo, azul-marinho,\\
encarnado, preto e castanho)[bis]\\
{eoc}\\
\\
De verde pintei a campina\\
sorridente ao sol da manhã\\
(amarela era a flor pequenina\\
qu'eu colhi para dar à mamã)[bis]\\
\\
[refrão]\\
\\
De azul pintei o mar sem fim\\
e de azul o céu vai aparecer\\
(encarnada era a rosa ao pé de mim\\
de tão linda nem a pude colher)[bis]\\
\\
[refrão]\\
\\
De preto pintei a noite escura\\
onde ninguém se quer aventurar\\
(castanha era a árvore madura\\
com tantos frutos a enfeitar)[bis]\\
\section{Livre (não há machado que corte)}

X: 1
M: 4/4
K: C
L: 1/8 
z2 CD   EG  FE | G2 D6 |z2DE  FA GF | G2E6 :|
w:Não há ma-cha-do que cor-te a ra-íz ao pen-sa-men-to
F2>F2  FF  FF | F2 E6 |G7D | F2 E6 ||
w:não há mor-te pa-rao ven-to  não  há mor-te
F2>F2  FF  FF | F2 E6 |G7B, | D2 C6 |]
w:não há mor-te pa-rao ven-to  não  há mor-te
%\includegraphics...
\\
(Não há machado que corte\\
a raíz ao pensamento) [bis]\\
(não há morte para o vento\\
não há morte) [bis]\\
\\
Se ao morrer o coração\\
morresse a luz que lhe é querida\\
sem razão seria a vida\\
sem razão\\
\\
Nada apaga a luz que vive\\
num amor num pensamento\\
porque é livre como o vento\\
porque é livre\\
\section{Para fazer uma canção de Natal}

X: 1
M: 4/4
K: Dm
Q: 1/4=60
L: 1/8 
"Dm" F2 DE FF DE| F2 AA F4 | "Gm" BA GF E2 D2 | BA GF E2 D2 | "A" ^C2 DD E2 ED |
w: Pa-ra fa-zer u-ma can-ção de Na-tal é pre-ci-so~a-çú-car, e pau de ca-ne-la lei-te~a-le-tri-a, pão
^CC DD E2 E2 | "Dm" DE FG AG FE | D2 D2 D2 D2 |]
w: se-co~e u-vas pas-sas e mis-tu-rar tu-do bem à luz das ve-las  
%\includegraphics...
\\
\\
_Para fazer uma canção de Natal           Dm\\
_é preciso açúcar, e pau de canela        Gm\\
_leite, aletria, pão seco e uvas passas   A\\
_e misturar tudo bem à luz das velas      Dm\\
\\
Flocos de neve para o refrão\\
e um grande pinheiro a brilhar no meio\\
uma galinha a pôr ovos dourados\\
farofa na voz para o recheio\\
\\
Casca de limão para o nariz\\
e vinho do Porto, para a garganta\\
música dos copos e dos talheres\\
o silêncio duma toalha branca.\\
\section{Embalo}

X: 1
M: 6/8
K: C
Q: 1/4=80
L: 1/8 
E | D2C D2C | B,2E E2G | F2E F2E | E2B B2c | 
w:Que le-ve le-v'éo meu me-nino Qu'o ven-to le-ve pra so-nhar qu'o
_B2A _B2A | G2F G2F | E2D _B2F | G2F F3 || 
w:so-nho pou-se de man-sinho Na man-sa luz do a-cor-dar
%\includegraphics...
\\
\\
Que leve, leve é o meu menino\\
Que o vento o leve pra sonhar\\
Que o sonho o pouse de mansinho\\
Na mansa luz do acordar\\
\\
Que acorde em cama bem quentinha\\
E quente enfie o seu roupão\\
E encontre a porta da cozinha\\
E encontre o doce, o leite, o pão\\
\section{Canção dos abraços}

X: 1
M: 2/4
K: Am
L: 1/8 
Q: 1/4=60
z E/2D/2 C>C | AA GG :| 
w:São dois bra-ços, são dois bra-ços
w:ser-vem pra dar um a-bra-ço
z A/2B/2 c>A | cA B>G | BG AE | AA GG |:  
w:a-ssim co-mo qua-tro bra-ços ser-vem pra dar dois a-bra-ços
z E/2D/2 C>C | AA GG :| 
w:Ea-ssim por aí fo-ra
w:té que quan-do for a ho-ra
z A/2B/2 c>A | cA B>G | BG AE | AA GG |:
w:vão ser tan-tos os a-bra-ços que não vão che-gar os bra-ços 
z A/2B/2 c>A | cA BG :|  
w:vão ser tan-tos os a-bra-ços
w:que não vão che-gar os bra-ços
z A/2G/2 AA5 ||
w:prós a-bra-ços 
%\includegraphics...
\\
\\
São dois _braços, _são dois _braços    C F G\\
servem _pra dar _um a_braço            C F G\\
assim _como quatro _braços             F G\\
servem _pra dar dois a_braços          F G\\
\\
E assim por aí fora\\
té que quando for a hora\\
vão ser tantos os abraços\\
que não vão chegar os braços\\
\\
Vão ser _tantos os a_braços            F G\\
que não _vão chegar os _braços         F G\\
prós a_braços                          Am\\
\section{A linda maça reineta}

X: 1
M: 6/8
K: D
L: 1/8 
DDD EDE | F3 F3 | FFF GFG | A6 | BBB dcB | A6 |
w:a lin-da ma-çã rei-ne-ta a-pa-nha-da no po-mar to-dos a queri-am co-mer
^GGG GFG | A5 A | G2 G FGF | E3 E2 A | G2 G FGF | E6
w: to-dos a queri-am pro-var mas um di-a q'ar-re-li-a de-sa-pare-ceu a ma-çã
DDD EDE | F3 G2 G | F2 F EDE | D6 |]
w:o Pe-dro diz não fui eu foi tal-vez a mi-nha ir-mã
%\includegraphics...
\\
\\
A linda maçã reineta\\
apanhada no pomar\\
todos a queriam comer\\
todos a queriam provar\\
\\
Mas um dia que arrelia \\
desapareceu a maçã\\
o Pedro diz não fui eu \\
foi talvez a minha irmã\\
\\
A irmã muito zangada\\
diz c'oa cabeça que não\\
eu não comi a maçã\\
quem comeu foi meu irmão\\
\\
Mas oh! que grande surpresa\\
toda a gente ria ria\\
o cão debaixo da mesa\\
contente a maçã roía\\
\section{A chuva}

X: 1
M: 3/4
K: C
L: 1/8
E2 | C2 CC EE | G2 G2 A2 | G2 GF ED | C2 z2 EE |
w: A chu-va é~um pin-gue pin-gue cons-tan-te e brin-ca-lhão pin-gue
C2 C2 EE | G2 G2 GA | G z2 F ED | C2 z2 cc | c2 c2 BA | 
w: ping-gue pin-gue ping-gue vai pin-gan-d~e cai no chão Mo-lha tu-do tu-do
G2 G2 FE | D2 z F ED | G2 z2 c2 | c2 cc BA | G2 G2 cc | B2 z B dd | c4 z2 |]
w: mo-lha mo-lha tu-do no jar-dim ea gen-te quan-do se mo-lha faz a-tchim a-tchim a-tchim
%\includegraphics...
\\
\\
A chuva é um pingue pingue \\
constante e brincalhão \\
pingue pingue pingue pingue \\
vai pingando e cai no chão\\
\\
Molha tudo tudo molha \\
molha tudo no jardim \\
e a gente quando se molha \\
faz atchim atchim atchim\\
\section{Era uma vez um cuco que não gostava de couves}

X: 1
M: 2/4
K: C
L: 1/8
CC CG, | CD EE | EE DC | DE CC |
w:E-ra u-ma vez um cu-co que não gos-ta-va de cou-ves
CC CG, | CD EE | EE DC | DE CC |
w:E-ra u-ma vez um cu-co que não gos-ta-va de cou-ves
z2 z C |: G/2G/2G EC | GG EC :|
w:e'o cu-co não quis co-mer as cou-ves
CC CG, | CD E2 | EE DC | DE C2 |]
w:e'es-ta-va sem-pre'a di-zer: cou-ves não hei-de'eu co-mer
%\includegraphics...
\\
\\
Era uma vez um cuco\\
que não gostava de couves\\
e estava sempre a dizer: \\
couves não hei-de eu comer\\
e estava sempre a dizer: \\
couves não hei-de eu comer\\
\\
mandaram chamar o pau\\
para vir bater no cuco\\
e o pau não quis bater no cuco\\
e o cuco não quis comer as couves\\
e estava sempre a dizer: \\
couves não hei-de eu comer\\
e estava sempre a dizer: \\
couves não hei-de eu comer\\
\\
mandaram chamar o fogo \\
para vir queimar o pau\\
e o fogo não quis queimar o pau\\
e o pau não quis bater no cuco\\
e o cuco não quis comer as couves\\
e estava sempre a dizer: \\
couves não hei-de eu comer\\
e estava sempre a dizer: \\
couves não hei-de eu comer\\
\\
mandaram chamar a água\\
para vir apagar o fogo\\
e a água não quis apagar o fogo\\
...\\
\\
mandaram chamar o boi\\
para vir beber a água\\
e o boi não quis beber a água\\
...\\
\\
mandaram chamar o homem\\
para vir matar o boi\\
e o homem não quis matar o boi\\
...\\
\\
\\
mandaram chamar o polícia\\
para vir prender o homem\\
e o polícia não quis prender o homem\\
...\\
\\
mandaram chamar a morte\\
para vir levar o polícia\\
e a morte ia levar o polícia\\
e o polícia já quis prender o homem\\
e o homem já quis matar o boi\\
e o boi já quis beber a água\\
e a água já quis apagar o fogo\\
e o fogo já quis queimar o pau\\
e o pau já quis bater no cuco\\
e o cuco já quis comer as couves\\
e não mais se ouviu dizer\\
couves não quero comer\\
\section{Lá vai o comboio}

X: 1
M: 2/4
K: C
L: 1/8
CD EC | D2 G,2 | DE FD | E2 C2 | \\
w: Lá vai o com-bo-io lá vai a'a-pi-tar
CC B,A,| G,G, FF| E2 D2| C4 |]
w:Lá vai o com-bo-io p'ra bei-ra do mar
%\includegraphics...
\\
\\
Lá vai o comboio\\
lá vai a'apitar\\
Lá vai o comboio\\
p'ra beira do mar\\
\\
p'ra beira do mar\\
p'ra beira do rio\\
e os passageiros\\
cheiinhos de frio\\
\\
cheiinhos de frio\\
cheios de calor\\
e os passageiros\\
a tocar tambor\\
\section{No alto da montanha}

X: 1
M: 3/4
K: C
L: 1/4
G, | G>F E/2F/2 | E D G,| G/2F/2 E D | C2 :|
w: No al-to da mon-ta-nha per-ti-nho lá do céu
w: ha-via um cas-te-li-nho a-on-de'um rei vi-veu
C D>E | C/2D/2 E>E | D/2E/2 F>F | E/2F/2 G2 |
w: de lá se via o céu se via a ter-ra ao lon-ge'o mar
C A>G | F/2E/2 F D| D G/2F/2 E | D C2 |]
w: no al-to da mon-ta-nha quem'me de-ra lá mo-rar
%\includegraphics...
\\
\\
No alto da montanha\\
pertinho lá do céu\\
havia um castelinho\\
aonde um rei viveu\\
\\
{soc}\\
de lá se via o céu \\
se via a terra\\
ao longe o mar\\
no alto da montanha\\
quem me dera lá morar\\
{eoc}\\
\\
O rei era bondoso\\
a raínha também\\
amigos do seu povo\\
a todos queriam bem\\
\section{Uma sardinha}

X: 1
M: 2/4
K: C
Q: 1/4=120
L: 1/8 
C2CCC2E2 ::
w:U-ma sar-di-nha
w:Du-as sar-di-nhas
w:Três - sar-di-nhas
CG2GAEDC2 ::
w: um pau e um ga-a-to
C2CCCE2 ::
w:que se me-te-ram
G3AEDC2 ::
w:num sa-pa-a-to
G,CCCCCE2 ::
w:a-xi-xi-xi-xi-ua-a
G,CCCCA,G,2 ::
w:a-ua-au-au-au-xi-xi
G,C2 CCCE2 ::
w:de la se-nho-ri-ta
G3AEDC2 ::
w:lu-i-si-i-ta :|
%\includegraphics...
\\
\\
Uma sardinha [bis]\\
duas sardinhas [bis]\\
três sardinhas [bis]\\
um pau e um gato [bis]\\
que se esconderam [bis]\\
num sapato [bis]\\
ah xixixixiuaua [bis]\\
ah uauauauaxixi [bis]\\
de la senhorita [bis]\\
(luisita) [bis]\\
\\
uma sardinha [bis]\\
duas sardinhas [bis]\\
três sardinhas [bis]\\
um pau e um gato [bis]\\
que se disputarm [bis]\\
te tal maneira [bis]\\
de se meterem [bis]\\
na banheira [bis]\\
ah xixixixiuaua [bis]\\
ah uauauauaxixi [bis]\\
de la senhorita [bis]\\
(luisita) [bis]\\
\section{Bola colorida}

X: 1
M: 3/4
K: Gm
L: 1/8 
D2 B2 AG | D2 B2 AG | D2 c2 BA | G6 ::
w:bo-la de sa-bão chei-a das cor's qu'o sol pin-tou
c3 c dc | BA G2 B2 | AG F2 A2 | G6 :|
w: Lá vai e-la a bri-lhar sem-pr'a bri-lhar lá vai
%\includegraphics...
\\
Bola de sabão\\
cheia das cores\\
que o sol pintou\\
\\
O vento a brincar\\
p'ró céu azul\\
a foi soprar\\
\\
Lá vai ela a brilhar\\
sempre a brilhar\\
lá vai\\
\section{O Menino}

X: 1
M: 2/4
K: C
L: 1/8 
Ec  B2 A2 G2>A2 B2 A4 :| 
w:pa-dre no-sso pe-que-ni-no 
w:quetem a cha-ve do me-ni-no
cB B2>A2 d2>c2 B2 A4 | Ec B A2  E G2>A2 B2 A6 |:
w:quem lha deu quem la da-ri-a foi S. Pe-dro San-ta Ma-ri-a
e2>f2 e2 c2 B2 A6 
e2>f2 e2 c2 A8 :|
%\includegraphics...
\\
\\
Meu Padre-n_osso pequenino_  Em D Em\\
que tem a chave _do men_ino     D Em\\
-Quem lha deu, quem lh_a daria_,  D Em\\
foi S.P_edro, Santa Mar_ia        D Em\\
\\
Cruzei montes, cruzei fontes,\\
que o pecado não encontre\\
nem de dia nem de noite\\
nem ao pino do meio dia\\
\\
Já os galos pretos cantam\\
já os anjos se alevantam\\
já o Senhor subiu à cruz\\
para sempre\\
Amen\\
Jesus\\
\section{Neve}

X: 1
M: 2/4
K: C
Q: 1/4=60
L: 1/8
EEGG|F2D2|CDEF|D4|
w: Cai a ne-ve bran-ca so-bre~a na-tu-reza 
FFEE|A2G2|GFED|C4|
w: E na ter-ra~in-tei-ra há paz e be-leza
%\includegraphics...
\\
_Cai a neve br_anca       C F\\
_sobre a natur_eza        C G\\
_e na t_erra int_eir_a    F C F C\\
_há paz _e bel_eza        C G C\\
\\
Em toda a doçura\\
é tudo em real\\
já é meia-noite\\
noite de Natal\\
\\
E uma estrelinha\\
que dá estranha luz\\
anuncia ao mundo\\
já nasceu Jesus\\
\section{Três galinhas a cantar}

X: 1
M: 2/4
K: C
L: 1/8
CC GG |AA G2 |FF EE| DD C2 |GG FF| EE ED |GG FF|EE ED |CC GG| AA G2 |FF EE| DD C2 |]
w: três ga-li-nhas a can-tar vão p'ro cam-po pas-se-ar. A da fren-te'é a pri-mei-ra lo-go'as ou-tras em car-rei-ra, vão as-sim a pas-se-ar os bi-chi-nhos pro-cu-rar
%\includegraphics...
\\
Três galinhas a cantar \\
vão p'ro campo passear;\\
a da frent'é a primeira\\
logo'as outras em carreira,\\
vão assim a passear\\
os bichinhos procurar\\
\section{Este linho é mourisco}

X: 1
M: 3/4
K: Dm
L: 1/8
G>G GG GE| F>E D2 D2 | F>F F2 E2 | D2 D4 |
w: es-te li-nho é mou-ris-co ea fi-ta de-le na-mo-ra
[GB]>[GB] [GB][GB] [GB][EG]| [FA]>[EG] [D2F2] [D2F2]| [FA]>[FA] [F2A2] [E2G2] | [D2F2] [D4F4]
w: quem da-qui não tem a-mo-res pe-gao cha-péu vá-seem-bo-ra
|:[GBd]>[GBd] [GBd][GBd] [GBd][EGc]| [FAd]>[EGc] [D2F2A2] [D2F2A2]| [FAd]>[FAd] [F2A2d2] [E2G2c2] | [D2F2A2] [D4F4A4] :|
w:Ai-a-li-o-lai-o-lai-la-lo-lé lai-a-ró meu bem
w: re-ga-la-teo meu a-mo-ri re-ga-la-tee pa-ssa bem 
%\includegraphics...
    \\
\\
Este linho é mourisco\\
e a fita dele namora\\
quem daqui não tem amores\\
pega o chapéu vá-se embora\\
\\
Ai-a-li-o-lai-o-lai-lalolé(?)\\
lai-a-ró meu bem\\
regala-te o meu amor\\
regala-te e passa bem \\
\\
O minha mãe dos trabalhos\\
para quem trabalho eu\\
trabalho mato meu corpo\\
não tenho nada de meu\\
\\
Mondadeiras lá de baixo\\
mondai o meu linho bem\\
não olheis para a portela\\
que a merenda logo vem\\
\section{O menino está dormindo}

X: 1
M: 4/4
K: C
L: 1/8 
GG | GE2FG2 A2 | B2 B4 A2 | FF2 A2 G2 F2 | E2 E4 :|
w: O me-ni-no es-tá dor-min-do Nas pa-lhi-nhas des-pi-di-nho
G2 | cG2 A B2 c2 | e2 A4 A2 | G G2 A G2 F2 | E2 E4 :|
w: Os an-jos lh'es-tão can-tan-do Por a-mor tão po-bre-zi-nho
%\includegraphics...
\\
\\
O menino está dormindo \\
Nas palhinhas despidinho\\
Os anjos lh'estão cantando\\
Por amor tão pobrezinho\\
\\
O menino está dormindo \\
Nos braços da virgem pura\\
Os anjos lh'estão cantando\\
Hossana lá na altura\\
\\
O menino está dormindo \\
Nos braços de São José\\
Os anjos lh'estão cantando\\
Gloria tibi Domine\\
\\
O menino está dormindo \\
Um sono de amor profundo\\
Os anjos lh'estão cantando\\
Viva o Salvador do Mundo\\
\\
\section{A minha saia velhinha}

X:1
M: 6/8
L: 1/8
K:C
G | e2e eff | eee ef>e | ddd ded | c3-c2G | [e2g][eg] [eg][fa][fa] | [eg][eg][eg] [eg][fa]>[eg] | [df][df][df] [df][eg][df] | [c3e] zgg | a2g A3Bc3[df] |
w: A mi-nha sai-a ve-lhi-nha 'stá to-da ro-ti-nha d'an-dar a bai-lar - a-go-ra te-nh'u-ma no-va fei-ti-nha na mo-da p'ra eu es-tri-ar -. Mi-nha mãe ca-sai-\-me ce-do, enquanto sou rapariga: que o milho ceifado tarde não dá palha nem espiga!
%\includegraphics...
\\
\\
A minha saia velhinha \\
Está toda rotinha \\
d'andar a bailar\\
\\
agora tenh'uma nova \\
feitinha na moda \\
p'ra eu estriar. \\
\\
Minha mãe casai-me cedo,\\
enquanto sou rapariga: \\
que o milho ceifado tarde \\
não dá palha nem espiga!\\
\\
O meu amor era torto\\
e eu mandei-o cavacar: \\
agora já tenho lenha \\
para fazer um jintar.                                                                   \\
\section{Oh oh meu menino}

X: 1
M: 11/8
K: C
Q: 1/4=60
L: 1/8
A2G2E FG GE E2 |
w: Oh oh, meu me-~-ni-~-no
A2G2F GE EC C2 |
w: Oh oh, meu a-~-mor-i
F2 EC C DF EG GF |
w: qu'as vo-~-ssas pa-~-la-~-vras
F2 EC C DC C4 |]
w: nos ma-~-tam com-~ dor
%\includegraphics...
\\
\\
{soc}\\
Oh oh meu menino\\
Oh oh meu amor\\
qu'as vossas palavras\\
nos matam com dor\\
{eoc}\\
\\
Filhos de pai rico\\
em bercinhos doirados\\
e só vós meu menino\\
em palhinhas deitado\\
\\
[refrão]\\
\\
[A] Senhora lavava\\
S. José estendia\\
e o menino chorava\\
com o frio que fazia\\
\\
[refrão]\\
htmlnota: \\
Esta música que ouvi na zona de Miranda do Douro, é por vezes tocada em\\
gaita de fole (tio Pascoal) (e penso que isto se reflecte na própria melodia).\\
\\
Disseram-me que era cantada quando dão o Menino a beijar no Natal.\\
\section{Cantar dos Reis (Donões, Montalegre)}

X: 1
M: 3/4
K: F
Q: 1/4=60
L: 1/4
C | A2 G | FAG | F3 | D2 F | E2 D | CED | C3 | A,2 C | A2A | AGA | B2 A | G2F | E2 D | CDE | F3 :]|
w: A-qui vem as três ro-si-nhas qua-tro ou cin-co ou se-is se~o se-nhor nos dá li-cen - sa vi-mos lhe can-tar os reis
%\includegraphics...
\\
\\
Aqui vem as três rosinhas\\
quatro ou cinco ou seis\\
se o senhor nos dá licensa\\
vimos lhe cantar os reis\\
\\
Os três reis do oriente\\
já chegaram a Belém\\
visitar o Deus Menino\\
que Nossa Senhora tem\\
\\
O menino está no berço\\
coberto c'o cobertor\\
eos anjinhos estão cantando\\
louvado sej'o Senhor\\
\\
O Senhor por ser Senhor\\
nasceu nos tristes palheiros\\
deixou cravos deixou rosas\\
deixou lindos travesseiros\\
\\
também deixou a abelhinha\\
abelhinha com o seu mel\\
para fazer um docinho\\
ao divino Emanuel\\
\\
Você diz que tem bom vinho\\
có có có\\
venha-nos dar de beber \\
rintintin \\
florin-tintin \\
traililairo\\
\\
\\
\section{Oh meu S. Bentinho}

X: 1
M: 2/4
K: C
Q: 1/4=60
L: 1/8
|: G, | C2 DE | F2 FF | FE DF | E3 :|
w: Oh meu São Ben-ti-nho de trás do hos-pi-tal
|: E | FE DC | D2 DF | ED CB, | C3 :|
w: Tu des-~ ta sa-ú-de a quem es-ta-va mal
%\includegraphics...
\\
(Oh meu São Bentinho \\
de trás do hospital)[bis]\\
(tu deste a saúde \\
a quem estava mal)[bis]\\
\\
A quem estava mal \\
e aos outros também\\
oh meu São Bentinho \\
para sempre amen\\
\\
Oh meu São Bentinho \\
de lado de lá da ponte\\
onde puseste o pé \\
nasceu uma fonte\\
\\
Oh meu São Bentinho \\
velinhas a arder\\
se as velas se apagarem\\
voltai-as a acender\\
\\
htmlnota:\\
Esta música é cantada por quem vem á pequena capela do São Bento detrás do \\
hospital de S. Marcos em Braga. Normalmente as romarias são à quinta feira \\
e há uma pessoa que dita o verso e depois todos os outros o repetem enquanto \\
vão andando.\\
\section{Verde Gaio}

X: 1
M: 2/4
K: C
Q: 1/4=80
L: 1/8
E/2E/2 |AA Bc |A G2 E/2E/2 |AA Bc |AG c>B |AG FE |D C2 C/2D/2| E>D EF| D C2 :|  
w: Hei-de can-tar hei-de rir-~ Hei-de can-tar hei-de rir-~ hei-de ser mui-to a-le-gre hei-de ser mui-to a-le-gre 
E/2E/2 |A/2A/2A/2A/2 A/2A/2B/2c/2 | A G2 E/2E/2 |A/2A/2A/2A/2 A/2A/2B/2c/2 | AG
|: c>B |AG FE |D C2 C/2D/2| E>D EF|1 DC :|2 D C2 |]
%\includegraphics...
\\
\\
Hei-de cantar hei-de rir [bis]\\
hei-de ser muito alegre [bis]\\
hei-de mandar a tristeza [bis]\\
para o demo que a leve [bis]\\
\\
Verde gaio verde gaio verde guito [bis]\\
agora é que vai a meio\\
o rapaz do casaquito\\
agora é que vai a meio\\
o rapaz do casaquito\\
\\
O meu amor quer que eu tenha [bis]\\
juizo capacidade [bis]\\
tenha ele que é mais velho [bis]\\
eu sou de menor idade [bis]\\
\\
verde gaio ...\\
\\
Sei um saco de cantigas [bis]\\
e mais uma saquetinha [bis]\\
quando as quero cantar [bis]\\
desato-lhe a baracinha [bis]\\
\section{A Ana quer}

X: 1
M: 2/4
K: C
Q: 1/4=60
L: 1/8 
z A cd | c4 |z A/2A/2 cd | cA/2c/2 dc/2A/2|  G4 |
z A cd | c4 |z A/2A/2 cd | cA/2c/2 dc|A2  G2 | 
z A/2c/2 A/2G/2F/2G/2 | AA/2c/2 A/2G/2F/2G/2 | AA/2c/2 AG | F4 |
z c/2A/2 G/2F/2G/2A/2| G4 |]
%\includegraphics...
\\
\\
A Ana quer\\
nunca ter saído da barriga da mãe\\
cá fora está-se bem\\
mas na barriga também era divertido\\
o coração ali à mão\\
os pulmões ali ao pé\\
ver como a mãe é\\
do lado que não se vê\\
\\
O que a Ana mais quer ser\\
quando for grande e crescer\\
é ser outra vez pequena\\
não ter nada que fazer\\
não ser pequena e crescer\\
de vez em quando nascer\\
e voltar a desnascer\\
a Ana quer\\
\\
htmlnota: Texto introdutório\\
\\
<p>Já era tarde. Os domingos passam muito depressa. Raul parara de chupar no dedo\\
e dormia e sorria. A mãe veio chamar o João e ele levou a bola. O sol a pouco \\
e pouco desapareceu atrás das casas e a rua encheu-se de sombras.\\
As pessoas corriam as persianas. O homem da loja enrolava o toldo.\\
Das janelas baixas vinha o barulho de conversas misturadas com o ruído\\
dos pratos dos talheres e o da televisão.</p>\\
<p>A Ana que tinha estado quase sempre calada durante toda a tarde, disse que \\
tinha frio.</p>\\
<p> A Ana quer nunca ter saído da barriga da mãe ...</p>\\
<p> A irmã mais velha da Ana pegou-lhe na mão e levou-a para casa. Virou-se para\\
trás olhou para o Pedro e disse-lhe -- Até amanhã.</p>\\
\section{Raúl tinha um Ioio}

X: 1
M: 2/4
K: C
Q: 1/4=60
L: 1/8 
dc Ad | dc Ad | dc Ad | GG AA | 
w:Ra-ul ti-nhaum i-oi-o que io-io-ia-va to-do o di-a
z/2 D/2E/2F/2 G>G | AG F2 |1 z/2 D/2E/2F/2 G>G | AG F2 :|2 z F/2A/2 GG | FE DD |]
w:quan-doo Ra-úl fa-zia ó-ó o i-o-io a-dor-me-cia
%\includegraphics...
\\
\\
Raúl tinha um ioio\\
que ioioiava todo o dia\\
quando o Raúl fazia ó-ó\\
o ioio adormecia\\
\\
E quando o Raúl chorava\\
porque o ó-ó não vinha\\
o ioio embalava\\
para baixo e para cima\\
\\
Raúl dormia e sonhava\\
e quando sonhava sorria\\
porque o io-io ioioiava\\
nos sonhos que Raúl via\\
\section{Vivam os pés}

X: 1
M: 3/4
K: F
L: 1/4
cGc| d2 c/2=B/2 | c G A | B2 A | G A B | G A B |G G F |1 G2 z :|2 G2 G |: F F
E| D2  D| CD  E |1 D D G:|2 D2 B | A G F | G A  B | A G F | G A  B | B/2c/2 c B
| A G F | F2 F | D3 |]
%\includegraphics...
\\
\\
Vivam os pés\\
que com pésinhos de lã\\
se põem a pé \\
logo pela manhã\\
\\
Os pés que vão \\
a pé para a escola\\
que pulam que bulem\\
que jogam à bola\\
\\
Que voltam pra casa\\
que vão aos recados\\
que à noite vão para cama\\
tão cansados\\
\\
Vivam os pés\\
que com pésinhos de lã\\
já adormeceram\\
e até amanhã\\
\section{Flor de Maio (?)}

X: 1
M: 2/4
K: C
Q: 1/4=80
L: 1/8
EE EG/2F/ | EE EG/2F/ | EE D/2C/B,/C/ | ED D2 ||
EE EG/2F/ | EE EG/2F/ | ED/2C/ B,/C/D/E/ | CC C2 ||
CC CB,/2C/ | ED D2 | DD DE/2F/ | E2 E2 ||
CC CB,/2C/ | ED D2 | G/2F/E/D/ C/B,/C/D/ | EC C2 :|
%\includegraphics...
\\
\section{Senhora das rosas}

X: 1
M: 6/8
K: C
L: 1/8
|:G | G3 GGG | F4 FF | E2 C D2 E | D3 D2 :|
w: Se-nho-ra quem cha-mais quan-do pas-so'ao vos-so la-do
w: Se-nho-ra quem o-lhai pon-do'os o-lhos no pas-sa-do
|:z| C3 D3 | F3 E3 | C3  D3 | D3 z2 :|
w: Não há ro-sas pra vos dar 
w: te-nho'al-guem por quem espe-rar
%\includegraphics...
\\
\\
Senhora quem chamais \\
quando passo ao vosso lado\\
senhora quem olhai pondo \\
os olhos no passado\\
\\
Senhora é bom esquecer\\
esse triste amor ardente\\
poois não sabeis viver\\
ao sabor do amor ausente\\
\\
Não há rosas pra vos dar\\
tenho alguém por quem esperar\\
\\
Senhora o meu amor\\
não vive num castelo\\
senhora o meu amor\\
é doutro olhar mais belo\\
\\
Eis as rosas que lhe vou dar\\
há mil rosas pra eu lhe dar\\
\section{Canário}

X: 1
M: 2/4
K: Dm
Q: 1/4=90
L: 1/8
A2|d2>d2| d2 ^ce | d2 Ac | c2>B2 | d2 c>B | A4 |
w: Es-ta ma-nhã fui à ca-ça lin-do ca-ná-rio ca-cei
z2 D2| A2>B2 | A2 GF | E2>F2 | G2 F2| E2 AG | F2 E2 | D4 |]
w: pa-ra tra-zer de pre-sen-te à fi-lha do* no-sso rei
%\includegraphics...
\\
\\
Esta manhã fui à caça\\
lindo canário cacei\\
para trazer de presente\\
à filha do nosso rei\\
\\
A filha do nosso rei\\
ela era brasileira\\
mandou fazer uma gaiola\\
da mais fininha madeira\\
\\
Depois da gaiola feita\\
seu canário meteu dentro\\
quer de dia quer de noite\\
era o seu divertimento\\
\\
Canário já se morreu\\
já lá vai para o deserto\\
diziam as moças todas\\
e morreu com o bico aberto\\
\\
Canário já se morreu\\
já lo vão ir a enterrar\\
diziam as moças todas\\
e morreu por confessar\\
\section{Achégate a mim, Maruxa}

X: 1
K: C
Q: 1/4=80
L: 1/8
CE DC DE F2 E2 z2 |
w: A-ché-ga-te~a mim, Ma-ru-xa
w: qué-ro-me ca-sar con-ti-go
CB, CD E3 C B,2 C2 z2 |
w: ché-ga-te ben, mo-re-ni-ña
w: se-rás mi-ña mu-lle-ri-ña
CB, CD E3 C B,2 A,2 z2 ::
w: ché-ga-te ben, mo-re-ni-ña
w: se-rás mi-ña mu-lle-ri-ña
Q: 1/4=120
C>B, C>B, A,B, CE D2 C>B, A,2 C>B, C>B, A,C B,2 A,2 z2:|
%\includegraphics...
\\
\\
Achégate a mim, Maruxa\\
chégate ben, moreniña\\
quérome casar contigo\\
serás miña mulleriña\\
\\
Adeus, estrela brilante\\
compañeiriña da lua\\
moitas caras teño visto\\
mais como a tua ningunha\\
\\
Adeus lubeiriña triste\\
de espaldas te vou mirando\\
non sei que me queda dentro\\
que me despido chorando\\
\section{Canção de embalar}

X: 1
M: 12/8 
K: Em
Q: 1/4=60
L: 1/8 
B>AG/2A/2 B>Gc/2B/2 A (d2 d3) | 
w: dor-me meu me-ni-no aes-tre-la-d'al-va
B>AG/2A/2 B>Gc/2B/2 A6 | 
w: já a pro-cu-rei e não a vi
d>cB/2d/2 c>BA/2c/2 B (G2 G3) |
w: see-la não vi-er de ma-dru-ga-da -
A>GF/2A/2 G>FG/2F/2 E6 | 
w: ou-tra queu sou-ber se-rá pra ti
e>Bc/2B/2 A>FG/2F/2 E6 |:
w: ou-tra qeu sou-ber se-rá pra ti
E/2G/2B/2G/2E/2B,/2D/2F/2A/2F/2D/2F/2 E6 :| 
%\includegraphics...
\\
\\
_Dorme meu menino a estrela d'_alva   Em D\\
_Já a procurei e não a _vi            Em D\\
Se ela não vi_er de madru_gada        Am Em\\
_Outra que eu sou_ber será p'ra _ti   Am h7 Em\\
_Outra que eu sou_ber será p'ra _ti   D  h7 Em \\
\\
Outra que eu souber na noite escura\\
Sobre o teu sorriso de encantar\\
Ouvirás cantando nas alturas\\
Trovas e cantigas de embalar\\
\\
Trovas e cantigas muito belas\\
Afina a garganta meu cantor\\
Quando a luz se apaga nas janelas\\
Perde a estrela d'alva o seu fulgor\\
\\
Perde a estrela d'alva pequenina\\
Se outra não vier para a render\\
Dorme qu'inda a noite é uma menina\\
Deixa-a vir também adormecer\\
\section{A formiga no carreiro}

X: 1
M: 6/8
K: C
Q: 1/4=160
L: 1/8 
A2A G2G | A2A G2G | A2A G2G |1 F2F E2E :|2 F2F E2|:G 
w:A for-mi-ga no ca-rrei-ro vi-nhaem sen-ti-do con-trá-rio
GAB c2G | GAB c2G | zGF E2E | D2D C2:| z | 
w:Ca-iu ao Te-jo Ca-iu ao Te-jo ao pé deum se-ptua-ge-ná-rio
z3 G3 | F3 E3 
|:z2 G | G2E A2G | G2 E :| c2G | G2F G2F | E2C :| 
w: Ler-pou tre-pou às tá-buas  que flu-tu-a-vam nas á-guas
zAA |A2F c2B | A3 A2A | G2G AGF | E2 G
w:e do ci-mo du-ma de-las vi-rou-se pro for-mi-guei-ro
|: GAB | c2 G GAB | c2 G2 GF |E2E D2D | C3 ::
w: mu-dem de ru-mo mu-dem de ru-mo já lá vem ou-tro ca-rreiro
zCD E2E | DED C2C |1 G,CC D2D | G,CD E3 :|2 G,G,G, D2D | G,G,G, C3 :|
%\includegraphics...
\\
\\
A formiga no carreiro\\
vinha em sentido contrário\\
Caiu ao Tejo\\
ao pé de um septuagenário\\
(v1 v2 v1 v2 v3 v3 v4 v3 v3 v4)\\
\\
Lerpou trepou às tábuas (bis)\\
que flutuavam nas águas (bis)\\
e do cimo de uma delas\\
virou-se para o formigueiro\\
mudem de rumo (bis)\\
já lá vem outro carreiro\\
\\
A formiga no carreiro\\
vinha em sentido diferente\\
caiu à rua\\
no meio de toda a gente\\
\\
buliu abriu as gâmbeas\\
para trepar às varandas\\
e do cimo de uma delas\\
...\\
\\
A formiga no carreiro\\
andava à roda da vida\\
caiu em cima\\
de uma espinhela caída\\
\\
furou furou à brava\\
numa cova que ali estava\\
e do cimo de uma delas\\
...\\
\section{Menino de oiro}

X: 1
M: 9/8 
K: C
Q: 1/4=80
L: 1/8 
c2c dcd c2c      | edc AG2 |:  
w: o meu me-ni-né doi-ro é doi-ro fi-no 
EGA cB2 AGE DC2  :|
w:não fa-çam ca-so qé pe-que-ni-no
f2f gfg f2f | agf dc2 
w:Ve-nham a-ves do céu Pou-sar de man-si-nho
cdf ed2 dcA G F2 :| 
w: Por so-bros om-bros do meu me-ni-no
%\includegraphics...
\\
O meu menino é d'oiro\\
É d'oiro fino\\
Não façam caso que é pequenino\\
O meu menino é d'oiro\\
D'oiro fagueiro\\
Hei-de levá-lo no meu veleiro.\\
\\
Venham aves do céu\\
Pousar de mansinho\\
Por sobre os ombros do meu menino\\
Do meu menino, do meu menino\\
Venha comigo venham\\
Que eu não vou só\\
Levo o menino no meu trenó.\\
\\
Quantos sonhos ligeiros \\
p'ra teu sossego\\
Menino avaro não tenhas medo\\
Onde fores no teu sonho\\
Quero ir contigo\\
Menino de oiro sou teu amigo\\
\\
Venham altas montanhas\\
Ventos do mar\\
Que o meu menino\\
Nasceu p'r'amar\\
Venha comigo venham\\
Que eu não vou só\\
Levo o menino no meu trenó.\\
\\
O meu menino é d'oiro\\
É d'oiro é de oiro fino ....\\
\\
Venham altas montanhas\\
Ventos do mar ....\\
\\
\section{Que amor não me engana}

X: 1
M:4/4
K:F
Q:1/4=70
L:1/4
DAAB | GA3 :| DGGA | FG3 | EGEA | A/2F/2 D3 |:
w:Qua-mor não men-ga-na Se dan-ti-ga cha-ma Mal vi-va-mar-gu - ra
w:Com su-a bran-du-ra
dfcA | G/2 c/2 d3 :| DAAB | GA3 | EGEA | A/2F/2 D3|]
w:Eas vo-zes em-bar - cam Quan-to mais sea-par-tam Mais seou-vo seu gri - to
w:Num si-lên-cioa-fli - to
%\includegraphics...
\\
\\
_Que amor não me en_ga_na     Dm C Dm\\
_Com a sua bran_du_ra         Dm C Dm \\
_Se de ant_iga _cha_ma        Dm C Am C\\
_Mal vive _a amar_gura         G Am Dm\\
\\
Duma mancha negra\\
Duma pedra fria\\
Que amor não se entrega\\
Na noite vazia\\
\\
_E as _vozes em_bar_cam        Dm C G Dm\\
Num si_lêncio a_fli_to         C G Dm\\
Quanto mais se a_par_tam       C Dm\\
_Mais se _ouve o seu _grito     C Am Dm\\
\\
Muito à flor das águas\\
Noite marinheira\\
Vem devagarinho\\
Para a minha beira\\
\\
Em novas coutadas\\
Junto de uma hera\\
Nascem flores vermelhas\\
Pela Primavera\\
\\
Assim tu souberas\\
Irmã cotovia\\
Dizer-me se esperas\\
O nascer do dia\\
\section{Tu gitana}

X: 1
M: 2/4
K: Dm
Q: 1/4=60
L: 1/8 
DEF2 FE2 ED/2E/2 F3 | 
w:Tu gi-ta-na q'a-di-vi - nhas
FGA2 Bc2 G/2B/2A4 | 
w:Me lo di-gas, poes no lo sê
cde2 Af2 ed/2c/2A3 | 
w:Se sa-ldre de-ss'a-ven-tu - ra
AGF2 GE>C F/2E/2D4 |
w:Ô si ne-la mo - ri - ré
DEF2 FE2 ED/2C/2 A,3 | 
w: Ô si ne-la per-co la vi - da
FGA2 AG2 D/2E/2F4 | 
w: Ô si ne-la tri-um-fa-re
Ac=B2 GF2 ED/2C/2 _B,3 | 
w:Tu gi-ta-na q'a-di-vi - nhas
DEF2 GEC F/2E/2D4 |
w:Me lo di-gas, poes que no lo sê
%\includegraphics...
\\
\\
Tu gi_tana que _adiv_inhas     Am Em Am\\
Me lo _digas, _poes no lo _sê  C  G  C\\
Se sal_dre des_sa aven_tura    G  F  Am\\
Ô si _nela _mori_ré            C  Em Am\\
Ô si _nela _perco la _vida     Am Em Am\\
Ô si _nela _triumfa_re         C  G  C\\
Tu gi_tana que _adi_vinhas     D  C  F\\
Me lo _digas, _poes no lo _sê  Am E  Am\\
\end{document}